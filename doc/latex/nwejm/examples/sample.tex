\providecommand{\eulersidentity}{%
  \begin{equation*}
    e^{i\pi}+1=0
  \end{equation*}
}
\providecommand{\sample}[1]{%
  %
  \newacronym{nwejm-\languagename}{nwejm}{North-Western European Journal of Mathematics}
  %
  \begin{abstract}
    \lipsum[1]
  \end{abstract}
  %
  \keywords{foo,bar,baz}
  %
  \msc{11B13,11B30,11P70}
  %
  \acknowledgments{Thanks to mum, daddy and all my buddies.}
  %
  \maketitle
  %
  \section*{Introduction}
  %
  \subsection{Citations tests}
  #1
  %
  \subsection{Cross-references tests}
  Cf. \vref{thm:bolzano-weierstrass-\languagename,rmk:euler-\languagename} \&
  \vref{eq:euler-\languagename} \& \vref{sec:first-numbered-\languagename}.
  %
  \subsection{Miscellaneaous}
  \begin{itemize}
  \item It has been proved in the \century{19} \aside{more than 100 years ago}
    that...
  \item This has been conceptualized in the \century{-3} \aside*{more than 2000
      years ago}.
  \item \acrshort{nwejm-\languagename} \ie{} \acrlong*{nwejm-\languagename}.
  \item \acrshort{nwejm-\languagename} \ie*{} \acrlong*{nwejm-\languagename}.
  \end{itemize}
  %
  \subsection{Acronyms tests}
  \begin{enumerate}
  \item The present article is published in the \gls{nwejm-\languagename}.
  \item Moreover, the present article is published in the \gls{nwejm-\languagename}.
  \end{enumerate}
  %
  \subsection{Theorems tests}
  \begin{theorem}[Bolzano–Weierstrass]\label{thm:bolzano-weierstrass-\languagename}
    A subset of $\R^n$ ($n\in\N^*$) is sequentially compact if and only if it is
    closed and bounded.
  \end{theorem}
  \begin{proof}[not that easy!]
    ...
  \end{proof}
  \begin{definition}
    In Cartesian space $\R^n$ with the $p$-norm $L_p$, an open ball is the set
    \[
      B(r)=\set{x\in \R^n}[\sum _{i=1}^n\left|x_i\right|^p<r^p]
    \]
  \end{definition}
  \begin{remark}[Euler's identity]\label{rmk:euler-\languagename}
    One of the most beautiful mathematical equation:
    \begin{equation}
      e^{i\pi}+1=0
    \end{equation}
  \end{remark}
  \begin{lemma*}[Zorn]
    Suppose a partially ordered set $P$ has the property that every chain has an
    upper bound in $P$. Then the set $P$ contains at least one maximal element.
  \end{lemma*}
  \begin{axiom}\label{my-axiom-\languagename}
    The following assertions are considered as true.
    \begin{assertions}
    \item\label{rare-expensive-\languagename} Anything that is scarce also is
      expensive.
    \item\label{cheap-horse-\languagename} A cheap horse is scarce.
    \end{assertions}
  \end{axiom}
  According to \vref{rare-expensive-\languagename,cheap-horse-\languagename}
  from \vref{my-axiom-\languagename}, a cheap horse is expensive.
  %
  \subsection{Dummy text and nice equation}
  %
  \lipsum[2-6]
  %
  \begin{equation}\label{eq:euler-\languagename}
    e^{i\pi}+1=0
  \end{equation}
  %
  \lipsum[8-15]
  %
  \section{First (numbered) section}\label{sec:first-numbered-\languagename}
  \lipsum[2]
  \subsection{First subsection}
  \lipsum[3-8]
  \subsection{Second subsection}
  \lipsum[9-15]
  \section{Second (numbered) section}
  \lipsum[16-38]
  \printbibliography
}
