\documentclass[dutch]{nwejmart}
\usepackage{lipsum}                    % Should'nt be used in a real article!
\addbibresource{sample.bib}            % Example of simple bibliographic file
\addbibresource{biblatex-examples.bib} % Example of sophisticated bibliographic file
%
\newacronym{nwejm-dutch}{nwejm}{North-Western European Journal of Mathematics}
%
\begin{document}
\title{Artikel Titel (dutch)}
\author[affiliation={Affiliation 8}]{Last8, First8}
\author[affiliation={Affiliation 9}]{Last9, First9}
\author[affiliation={Affiliation 10}]{Last10, First10}
\author[affiliation={Affiliation 11}]{Last11, First11}
\author[affiliation={Affiliation 12}]{Last12, First12}
\author[affiliation={Affiliation 13}]{Last13, First13}
\author[affiliation={Affiliation 14}]{Last14, First14}
\author[affiliation={Affiliation 15}]{Last15, First15}
\author[affiliation={Affiliation 16}]{Last16, First16}
\author[affiliation={Affiliation 17}]{Last17, First17}
\author[affiliation={Affiliation 18}]{Last18, First18}
\author[affiliation={Affiliation 19}]{Last19, First19}
\author[affiliation={Affiliation 20}]{Last20, First20}
%
\begin{abstract}
  \lipsum[1]
\end{abstract}
%
\keywords{foo,bar,baz}
%
\msc{11B13,11B30,11P70}
%
\acknowledgments{Thanks to mum, daddy and all my buddies.}
%
\maketitle
%
\section*{Recommendations for \LaTeX}
Don't use:
\begin{itemize}
\item \verb+$$...$$+ but \verb+\[...\]+ or \verb+\begin{equation}...\end{equation}+
\item \verb+$a \over b$+ but \verb+$\frac{a}{b}$+
\item \verb+{\cal ...}+ but \verb+\mathcal{...}+
\item \verb+{\bf ...}+ but \verb+\textbf{...}+
\item \verb+{\it ...}+ but \verb+\emph{...}+
\item \verb+\'e+ for instance to get an accent but type it directly (\verb+é+)
  using the UTF8 encoding.
\end{itemize}
More generally, it is worth having a look at documents that highlight obsolete
commands and
packages\autocite{ensenbach2016,ensenbach2011,trettin2007,ensenbach2011a,trettin2007a}.
%
\section*{Introduction}
%
\subsection{Citations tests}
\begin{enumerate}
  \item It\footnote{Foo bar.} is well known\autocite{gonzalez}
    that... Moreover, it is well known\autocite{iliad} that...
  \item \textcite{gonzalez} have proved... Moreover, \textcite{iliad}
    have proved...
\end{enumerate}
%
\subsection{Cross-references tests}
Cf. \vref{thm-bolzano-weierstrass-\languagename,rmk-euler-\languagename} \&
\vref{eq-euler-\languagename} \& \vref{sec-first-numbered-\languagename}.
%
\subsection{Miscellaneaous}
\begin{itemize}
\item It has been proved in the \century{19} \aside{more than 100 years ago}
  that...
\item This has been conceptualized in the \century{-3} \aside*{more than 2000
    years ago}.
\item \acrshort{nwejm-\languagename} \ie{} \acrlong*{nwejm-\languagename}.
\item \acrshort{nwejm-\languagename} \ie*{} \acrlong*{nwejm-\languagename}.
\end{itemize}
%
\subsection{Acronyms tests}
\begin{enumerate}
\item The present article is published in the \gls{nwejm-\languagename}.
\item Moreover, the present article is published in the \gls{nwejm-\languagename}.
\end{enumerate}
%
\subsection{Theorems tests}
\begin{theorem}[Bolzano–Weierstrass]\label{thm-bolzano-weierstrass-\languagename}
  A subset of $\bbR^n$ ($n\in\bbN^*$) is sequentially compact if and only if it is
  closed and bounded.
\end{theorem}
\begin{proof}[not that easy!]
  ...
\end{proof}
\begin{definition}
  In Cartesian space $\bbR^n$ with the $p$-norm $L_p$, an open ball is the set
  \[
    B(r)=\set{x\in \bbR^n}[\sum _{i=1}^n\left|x_i\right|^p<r^p]
  \]
\end{definition}
\begin{remark}[Euler's identity]\label{rmk-euler-\languagename}
  One of the most beautiful mathematical equation:
  \begin{equation}
    \E[\I\pi]+1=0
  \end{equation}
\end{remark}
\begin{lemma*}[Zorn]
  Suppose a partially ordered set $P$ has the property that every chain has an
  upper bound in $P$. Then the set $P$ contains at least one maximal element.
\end{lemma*}
\begin{axiom}\label{my-axiom-\languagename}
  The following assertions are considered as true.
  \begin{assertions}
  \item\label{rare-expensive-\languagename} Anything that is scarce also is
    expensive.
  \item\label{cheap-horse-\languagename} A cheap horse is scarce.
  \end{assertions}
\end{axiom}
According to \vref{rare-expensive-\languagename,cheap-horse-\languagename}
from \vref{my-axiom-\languagename}, a cheap horse is expensive.
%
\subsection{Dummy text and nice equation}
%
\lipsum[2-6]
%
\begin{equation}\label{eq-euler-\languagename}
  \E[\I\pi]+1=0
\end{equation}
%
\lipsum[8-15]
%
\section{First (numbered) section}\label{sec-first-numbered-\languagename}
\lipsum[2]
\subsection{First subsection}
\lipsum[3-8]
\subsection{Second subsection}
\lipsum[9-15]
\section{Second (numbered) section}
\lipsum[16-38]
\printbibliography
%
\end{document}
