\documentclass[french,nolocaltoc]{nwejmart}
\usepackage{denisbdoc}
\usepackage{tabulary}
\usepackage[firstpage]{draftwatermark}
%
\indexsetup{level=\section*,toclevel=section,noclearpage}
\makeindex[columns=2]
\tcbset{index format=off}
%
\lstMakeShortInline[style=dbtex]|
\tcbset{listing options={escapechar=+}}
%
\SetWatermarkColor{gray!10}%
\SetWatermarkFontSize{20mm}%
\SetWatermarkText{\MakeUppercase{travail en cours}}%
%
\addbibresource{nwejm-fr.bib}
%
\newacronym{nwejm}{nwejm}{North-Western European Journal of Mathematics}
\newacronym{bap}{bap}{bounded approximation property}
\newacronym{msc}{msc}{Mathematical Subject Classification}
\newacronym{url}{url}{Uniform Resource Locator}
%
\newtheorem{article}
\newtheorem[title=Fait,style=definition]{fact}
%
\DeclareMathOperator{\cat}{Cat}
\newcommand{\mybinop}{\mathbin{@}}
%
\begin{document}
\title[Documentation de la classe \LaTeX{} du
\texorpdfstring{\nwejm}{NWEJM}]{Documentation de la classe \LaTeX{} du \nwejm*{}
  destinée aux auteurs}
%
\author[%
affiliation={%
  Université du Littoral Côte d'Opale, Laboratoire de mathématiques pures et
  appliquées (\nolinkurl{denis.bitouze@lmpa.univ-littoral.fr})%
},%
]{Bitouzé, Denis}
%
\maketitle
%
\etocarticlestylenomarks
\localtableofcontents
%
\DeclareFixedFootnote{\conseilslatex}{Et donne de façon subliminale quelques conseils
  \LaTeX{}. On consultera \autocite{Ensenbach} pour un panorama plus complet des
  commandes et extensions obsolètes (et autres erreurs).}
\DeclareFixedFootnote{\suitecompilations}{Pour automatiser ces compilations, on
  pourra recourir au \enquote{compilateur} \program{latexmk}, fourni par toute
  distribution \TeX{} moderne, en s'aidant du fichier de configuration
  \file{latexmkrc} joint à la présente classe.}

\section{Introduction}
\label{sec-introduction}

\begin{dbremark}{Documentation : travail en cours}{}
  La présente documentation est un travail en cours. Nous prions les lecteurs
  qui ont des remarques à formuler ou qui trouveront des erreurs de bien vouloir
  nous les signaler à l'adresse courriel figurant en page de titre.
\end{dbremark}

La classe \LaTeX{} \nwejmauthor{} est destinée aux auteurs souhaitant publier un
article dans le \href{http://math.univ-lille1.fr/~nwejm/}{\gls{nwejm}}. Cette
classe a pour but :
\begin{enumerate}
\item de reproduire fidèlement la maquette de la \nwejm{}, permettant ainsi
  aux auteurs de pouvoir travailler la mise en page de leur document dans des
  conditions réelles ;
\item de fournir un certain nombre d'outils (commandes et environnements)
  destinés à faciliter la rédaction de documents, notamment ceux contenant des
  formules de mathématiques.
\end{enumerate}

\begin{dbwarning}{Codage d'entrée \protect\lstinline+UTF-8+ obligatoire}{}
  La \nwejmauthorcl{} du \nwejm{} repose de façon cruciale sur un codage
  d'entrée en |UTF-8| :
  \begin{itemize}
  \item elle charge le \Package{inputenc}\footnote{%
      Il convient donc de \emph{ne pas} charger ce package : on se reportera
      à la \vref{sec-canevas-de-fichier} pour un canevas de fichier source
      destiné à la \nwejm{} et notamment pour un préambule typique.%
      %
    }
    %
    avec l'option \docAuxKey{utf8} ;
  \item elle suppose que les fichiers sources \file{.tex} sont effectivement
    codés en |UTF-8| : on veillera donc à ce que l'éditeur utilisé soit réglé
    sur ce codage d'entrée.
  \end{itemize}
\end{dbwarning}

\begin{dbwarning}{Format et compilation \hologo{pdfLaTeX}}{}
  La composition des numéros du \nwejm*{} met en œuvre :
  \begin{itemize}
  \item le format \hologo{LaTeX}\footnote{Sous sa version \hologo{LaTeXe}, la
      plus courante de nos jours.}. On évitera donc les commandes des autres
    formats tels que \hologo{plainTeX} ;
  \item le compilateur privilégié est \hologo{pdfLaTeX}. En particulier, le
    compilateur \hologo{LaTeX} n'étant \emph{a priori} pas utilisé, nous
    préférons :
    \begin{enumerate}
    \item comme formats d'images : \format{jpg}, \format{png} et
      \format{pdf}\footnote{Le format recommandé est \format{pdf} car il est
        vectoriel si bien que la qualité des images n'est pas dégradée en cas de
        redimensionnement.} ;
    \item éviter l'usage de code \program{PostScript}, notamment via
      \program{PSTricks} et, comme outil de dessin, privilégions par exemple le
      \Package{TikZ}.
    \end{enumerate}
  \end{itemize}
\end{dbwarning}

\section{Configuration de l'article}

\subsection{Langue de l'article}
\label{sec:langue-de-larticle}

Le \nwejm{} accepte des articles dans chacune des quatre langues
\begin{itemize}
\item anglais ;
\item français ;
\item allemand ;
\item néerlandais ;
\end{itemize}
et la \nwejmauthorcl{} permet de spécifier la langue choisie\footnote{La gestion
  des langues est assurée en sous-main par le \Package{babel}.} au moyen des
options de classe suivantes.
% \refKey{english}, \refKey{french}, \refKey{ngerman} \refKey{dutch}

\begin{docKey}{english}{}{pas de valeur, option par défaut}
  Cette option définit l'anglais comme langue de l'article.
\end{docKey}
\begin{docKey}{french}{}{}
  Cette option définit le français comme langue de l'article.
\end{docKey}
\begin{docKey}{german}{}{}
  Cette option définit l'allemand comme langue de l'article. Elle a pour alias
  \docAuxKey{ngerman}.
  \begin{dbremark}{Option de \package*{babel} sous-jacente à l'option
      \protect\refKey{german}}{}
    L'option de langue \refKey{german} de la \nwejmauthorcl{} appelle en
    sous-main l'option \docAuxKey*{ngerman} du \Package*{babel}.
  \end{dbremark}
  \begin{docKey}{dutch}{}{}
  Cette option définit le néerlandais comme langue de l'article.
\end{docKey}
\end{docKey}

\begin{dbremark}{Langue par défaut de l'article}{}
  Si aucune de ces options n'est spécifiée, la langue de l'article est par
  défaut l'anglais.
\end{dbremark}

\subsection{Autres configurations}
\label{sec:autr-conf}

\begin{docCommand}{articlesetup}{\marg{options}}
  Cette commande permet de configurer les \meta{options} de l'article en cours
  sous la forme clés/valeurs. La seule clé fournie pour l'instant est
  \refKey{gradient} (mais ceci peut être amené à changer).
\end{docCommand}

\begin{dbwarning}{Commande \protect\refCom{articlesetup} : pas en préambule !}{}
  La commande \refCom{articlesetup} doit être utilisée \emph{exclusivement} dans
  le corps du document (de préférence juste après |\begin{document}|) sans quoi
    la configuration choisie, certes prise en compte dans l'article isolé,
    risque de ne pas l'être dans le volume complet !
\end{dbwarning}

\section{Préparation du \enquote{titre} de l'article}
\label{sec-prep-de-lart}

Cette section liste les commandes, options et environnement permettant de
\emph{préparer} le \enquote{titre} de l'article ainsi que son éventuelle partie
finale.

\subsection{Titre, sous-titre}
\label{sec-titre}

\begin{docCommand}[doc description=\mandatory]{title}{\oarg{titre court}\marg{titre}}
  Cette commande définit le \meta{titre} de l'article qui apparaît :
  \begin{itemize}
  \item en début d'article ;
  \item en titre courant\footnote{En entête.} sur les pages paires ;
  \item comme métadonnée \enquote{Titre} du fichier \textsc{pdf} correspondant ;
  \item dans la table des matières du volume où sera publié l'article.
  \end{itemize}
\end{docCommand}

\begin{docCommand}{subtitle}{\oarg{sous-titre court}\marg{sous-titre}}
  Cette commande définit l'éventuel \meta{sous-titre} de l'article. Celui-ci
  est automatiquement ajouté au titre.
\end{docCommand}

Les commandes \refCom{title} et \refCom{subtitle} admettent un argument
optionnel permettant de spécifier un \meta{(sous-)titre court} de l'article qui
figure alors à la place du (sous-)titre \enquote{normal} en titre courant et
dans la table des matières.

\begin{bodycode}[listing options={deletekeywords={title,subtitle}}]
\title[Le Traité du calcul des probabilités]{Le Traité du calcul des
  probabilités et de ses applications}
\subtitle[Étendue et limites d'un projet borélien]{Étendue et limites
  d'un projet borélien de grande envergure (1921-1939)}
\end{bodycode}

\begin{dbremark}{Affichage des titre et sous-titre}{}
  Pour que les titre et sous-titre soient affichés, il est nécessaire de
  recourir à la commande habituelle \refCom{maketitle}.
\end{dbremark}

\subsection{Auteur(s)}
\label{sec-auteurs}

Un auteur d'article est spécifié au moyen de la commande \refCom{author}
suivante. En cas d'auteurs multiples, il suffit de (et il faut) recourir
à plusieurs occurrences de cette commande.

\begin{docCommand}{author}{\oarg{options}\brackets{\meta{Nom}, \meta{Prénom}}}
  Cette commande, facultative, définit un auteur d'article.
\end{docCommand}
%
\begin{dbwarning}{Format des prénom et nom de l'auteur}{}
  On veillera à ce que :
  \begin{enumerate}
  \item la saisie des prénom et nom de l'auteur soit conforme à la syntaxe
    (identique à celle de \hologo{BibTeX} et \package{biblatex}) :
    %
\begin{bodycode}[listing options={showspaces}]
"\meta{Nom}", "\meta{Prénom}"
\end{bodycode}
    %
  \item les éventuels caractères diacritiques (accents, etc.) figurent dans les
    \meta{Prénom} et \meta{Nom} ;
  \item le \meta{Nom} \emph{ne} soit \emph{pas} saisi en capitales (sauf pour la
    ou les majuscules)\footnote{Dans les articles en français, ce nom sera
      automatiquement composé en petites capitales.}.
  \end{enumerate}
\end{dbwarning}

La commande \refCom{author} admet un argument optionnel permettant de spécifier
la ou les affiliations de l'auteur au moyen d'une des clés \refKey{affiliation}
et \refKey{affiliationtagged} suivantes.

\begin{docKey}{affiliation}{={\oarg{tag}\marg{affiliation}}}{pas de valeur par défaut,
    initialement vide}
  Cette clé permet de définir une (unique) \meta{affiliation}. Pour un auteur
  ayant plusieurs affiliations, cette clé est utilisée autant de fois que
  nécessaire.

  En outre, chaque affiliation peut recevoir un \meta{tag} en vue d'être
  réutilisée pour un autre auteur.
\end{docKey}

\begin{docKey}{affiliationtagged}{={\marg{tag}}}{pas de valeur par défaut,
    initialement vide}
  Cette clé permet de spécifier une (unique) affiliation déjà définie et taguée
  pour un auteur précédent.
\end{docKey}

% \begin{bodycode}
% % Exemple où 2 auteurs partagent l'affiliation "\meta{affiliation$_3$}".
% \author[%
%   affiliation="\marg{affiliation$_1$}",%
%   affiliation="\marg{affiliation$_2$}",%
%   affiliation="\oarg{tag de l'affiliation$_3$}\marg{affiliation$_3$}",%
% ]{"\meta{Nom$_1$}", "\meta{Prénom$_1$}"}
% \author[%
%   affiliationtagged="\oarg{tag de l'affiliation$_3$}",%
% ]{"\meta{Nom$_2$}", "\meta{Prénom$_2$}"}
% \end{bodycode}

\begin{bodycode}[listing options={deletekeywords={author}}]
\author[
  affiliation={Laboratoire \textsc{sphere}, Université Paris Diderot}
  ]{Bustamante, Martha-Cecilia}
\author[
  affiliation=[aff2]{\textsc{lpma}, Université Pierre et Marie Curie},
  affiliation={\textsc{ghdso}, Université Paris-Sud}]{Cléry, Matthias}
\author[
  affiliationtagged={aff2}
]{Mazliak, Laurent}
\end{bodycode}

\begin{dbwarning}{Paires d'accolades
    % des clés \refKey{affiliation} et \refKey{minibio}
    (relativement) obligatoires}{}
  Au cas (probable) où une valeur \meta{affiliation} contienne une ou plusieurs
  virgules, la paire d'accolades l'entourant est obligatoire.
\end{dbwarning}

\begin{dbwarning}{Affichage des prénoms et noms du ou des auteurs}{}
  Pour que les prénoms et noms du ou des auteurs soient affichés, il est
  nécessaire :
  \begin{enumerate}
  \item de recourir à la commande habituelle \refCom{maketitle} ;
  \item de procéder à une suite de compilations\suitecompilations{} :
    \begin{enumerate}
    \item \hologo{pdfLaTeX} ;
    \item \hologo{biber} ;
    \item \hologo{pdfLaTeX}.
    \end{enumerate}
  \end{enumerate}
\end{dbwarning}

\subsection{Résumé}
\label{sec-resume}

\begin{docEnvironment}[doc description=\mandatory,doclang/environment content=résumé]{abstract}{}
  Cet environnement est destiné à recevoir le \meta{résumé} de l'article.
\end{docEnvironment}

\subsection{Mots clés}

\begin{docCommand}[doc description=\mandatory]{keywords}{\oarg{variante des mots clés}\marg{mots clés}}
  Cette commande permet de spécifier les \meta{mots clés} de l'article sous la
  forme d'une liste séparée par des virgules.

  La propriété \enquote{\texttt{pdfkeywords}} du \File{PDF} produit contient
  automatiquement les \meta{mots clés} spécifiés. Dans le cas où ceux-ci
  contiennent des caractères non admis en propriétés des fichiers \file{PDF}
  (notamment des mathématiques), on spécifiera en argument optionnel une
  \meta{variante des mots clés} ne contenant que des caractères autorisés.
\end{docCommand}

\begin{bodycode}[listing options={deletekeywords={keywords}}]
\keywords[N\string_p-space]{$\mathcal{N}_p$-space}
\end{bodycode}

\subsection{\texorpdfstring{\acrfull{msc}}{MSC}}

\begin{docCommand}[doc description=\mandatory]{msc}{\marg{\acrshort{msc}}}
  Cette commande permet de spécifier le (ou les) \meta{\acrshort{msc}} de
  l'article sous la forme d'une liste séparée par des virgules.
\end{docCommand}

\section{Production du titre de l'article}
\label{sec-creation-du-titre}

Le titre proprement dit de l'article, regroupant tous les éléments saisis à la
\vref{sec-prep-de-lart}, est produit par la commande standard
\refCom{maketitle}.

\begin{docCommand}[doc description=\mandatory]{maketitle}{}
  Cette commande \emph{produit} l'affichage du \enquote{titre} de l'article,
  c'est-à-dire :
  \begin{itemize}
  \item son titre et son éventuel sous-titre (commandes \refCom{title} et
    \refCom{subtitle}) ;
  \item son ou ses auteurs, sous la forme de leurs noms et prénoms (commande(s)
    \refCom{author}) et, en note de bas de page, leurs affiliations respectives ;
  \item son éventuel résumé (environnement \refEnv{abstract}) ;
  \item son ou ses mots clés (commande \refCom{keywords}) ;
  \item son ou ses \acrshort{msc} (commande \refCom{msc}).
  \end{itemize}
\end{docCommand}

\begin{bodycode}[listing options={deletekeywords={title,subtitle,author,keywords},deletekeywords={[2]{first}}}]
\title[Le Traité du calcul des probabilités]{Le Traité du calcul des
  probabilités et de ses applications}
\subtitle[Étendue et limites d'un projet borélien]{Étendue et limites
  d'un projet borélien de grande envergure (1921-1939)}
%
\author[
  affiliation={Laboratoire \textsc{sphere}, Université Paris Diderot}
  ]{Bustamante, Martha-Cecilia}
\author[
  affiliation=[aff2]{\textsc{lpma}, Université Pierre et Marie Curie},
  affiliation={\textsc{ghdso}, Université Paris-Sud}]{Cléry, Matthias}
\author[
  affiliationtagged={aff2}
]{Mazliak, Laurent}
%
\begin{abstract}
  Cet article est consacré à l'étude détaillée du vaste projet [...]
\end{abstract}
%
\keywords{Probabilités, statistiques, balistique, assurance, jeux,
  Émile Borel, Institut Henri Poincaré}
\msc{01A60, 01A74, 60-03, 60A05, 62-03}
%
\maketitle
\end{bodycode}

\section{Remerciements}
\label{sec:remerciements}

\begin{docCommand}{acknowledgements}{\marg{remerciements}}
  Cette commande, facultative, permet de spécifier des \meta{remerciements} pour
  un article.
\end{docCommand}

\begin{bodycode}[listing options={deletekeywords={author},deletekeywords={[2]{first,and}}}]
\acknowledgments{%
  The first author's research was supported by the Hungarian National
  Science Foundation Grants K81658 and K104183.  Research conducted
  while the second author enjoyed the hospitality of the Alfréd Rényi
  Institute of  Mathematics, and benefited from the \textsc{otka}
  grant K109789.%
}
\end{bodycode}

\begin{dbremark}{Affichage des remerciements sur les auteurs}{}
  Pour que les remerciements soient affichés, il est nécessaire de recourir à la
  commande \refCom{printbibliography} destinée à être utilisée en fin d'article.
\end{dbremark}

\section{Bibliographie}
\label{sec-bibliographie}

Pour composer les bibliographies, la \nwejmauthorcl{} recourt aux outils
modernes que sont le \Package*{biblatex} et le moteur \hologo{biber} dont
l'usage peut être brièvement résumé comme suit\autocite[Pour plus de détails,
cf. par exemple][en français]{Bitouze}.
\begin{enumerate}
\item Construire, dans un \meta{fichier bibliographique}\footnote{De tels
    fichiers sont d'extension \file{.bib}.}, une base bibliographique.

  Nous conseillons pour ce faire de recourir à un logiciel dédié, par exemple
  \href{http://jabref.sourceforge.net/}{\program{JabRef}} à configurer en se
  rendant dans le menu \menuentry{Options>Preferences} et, dans l'onglet :
  \begin{itemize}
  \item \menuentry{Général}\footnote{On pourra en profiter pour changer la langue de
      l'interface dans la liste déroulante \menuentry{Language}.}, en choisissant
    \menuentry{UTF8} dans la liste déroulante \menuentry{Default encoding} ;
  \item \menuentry{Advanced}, en cochant l'option \menuentry{BibLaTeX mode} ;
  \end{itemize}

  On n'oubliera pas de donner à chaque référence bibliographique une \meta{clé}
  permettant de l'identifier de façon unique dans la base
  bibliographique\footnote{Sous \program{JabRef}, on pourra se faire aider pour
    cela par l'icône en forme de \enquote{baguette magique} ou de clé.}.
\item Dans le fichier source \file{.tex} :
  \begin{enumerate}
  \item \emph{En préambule}, utiliser la commande \docAuxCommand{addbibresource}
    pour indiquer le \meta{fichier bibliographique} à considérer :
\begin{preamblecode}
\addbibresource{"\meta{fichier bibliographique}".bib}
\end{preamblecode}
  \item Dans le corps du document, utiliser la commande
    \docAuxCommand{autocite}\footnote{De préférence à la commande
      \docAuxCommand{cite}.}\footnote{La référence est alors détaillée en note
      de bas de page.}  ou, éventuellement, \docAuxCommand{textcite}\footnote{La
      référence est alors affichée au fil du texte, mais avec moins de détails.}
    du \Package*{biblatex} pour citer les références bibliographiques (pourvue
    d'une \meta{clé}) :
\begin{bodycode}
\autocite{"\meta{clé}"}
\textcite{"\meta{clé}"}
\end{bodycode}
  \item À la fin du document, faire figurer la liste des références
    bibliographiques au moyen de la commande \refCom{printbibliography}.
  \end{enumerate}
\item Procéder à une suite de compilations \hologo{pdfLaTeX}, \hologo{biber} et
  \hologo{pdfLaTeX}\suitecompilations.
\end{enumerate}

\begin{docCommand}{printbibliography}{}
  Cette commande produit la liste des références bibliographiques saisies selon
  la syntaxe du \Package*{biblatex}. Elle a été redéfinie de sorte
  à automatiquement afficher, avant la liste des références, les éventuels
  remerciements (commande \refCom{acknowledgements}).
\end{docCommand}

Par exemple, si le \File{.bib} contient :

\begin{bodycode}[listing only,listing options={morekeywords={Book,Article}}]
@Book{            har,
  author	= {Hartshorne, Robin},
  title		= {Algebraic geometry},
  note		= {Graduate Texts in Mathematics, No. 52},
  publisher	= {Springer-Verlag},
  address	= {New York},
  year		= {1977}
}
@Book{		  laz1,
  author	= {Lazarsfeld, Robert},
  title		= {Positivity in algebraic geometry. {I}},
  volume	= {48},
  note		= {Classical setting: line bundles and linear series},
  publisher	= {Springer-Verlag},
  address	= {Berlin},
  year		= {2004}
}
@Article{         shin,
  author	= {Shin, Kil-Ho},
  title		= {{$3$}-dimensional {F}ano varieties with canonical
                  singularities},
  journal	= {Tokyo J. Math.},
  volume	= {12},
  year		= {1989},
  number	= {2},
  pages	= {375-385}
}
\end{bodycode}

alors, le source \file{.tex} suivant :
\begin{bodycode}[listing only,listing options={deletekeywords={[1]{theorem},[2]{first,and,see}}}]
The first assertion is a direct consequence of the Riemann-Roch
formula for threefolds\autocite[See e.g.][437]{har} and
Kawamata-Viehweg vanishing\autocite[theorem~4.3.1]{laz1}. For the
second assertion, see \textcite[theorem~(0.4)]{shin}.
%
\printbibliography
\end{bodycode}

donne le texte figurant dans l'encadré intitulé \enquote{Exemple de
  bibliographie} \vpageref{biblio}.

\begin{refsegment}
  \ExplSyntaxOn
  \makeatletter
  \defbibheading{origsectionnotintoc}[References]{%
    \__nwejm_section*{#1}%
  }
  \makeatother
  \ExplSyntaxOff
\begin{bodycode}[text only,float,title={Exemple de bibliographie},label=biblio]
The first assertion is a direct consequence of the Riemann-Roch formula for
threefolds\autocite[See e.g.][437]{har} and Kawamata-Viehweg
vanishing\autocite[theorem~4.3.1]{laz1}.  For the second assertion, see
\textcite[theorem~(0.4)]{shin}.
%
\printbibliography[heading=origsectionnotintoc,segment=1]
\end{bodycode}
\end{refsegment}

\section{Commandes propres aux mathématiques}
\label{sec:comm-propr-aux}

La \nwejmauthorcl{} :
\begin{itemize}
\item charge les packages :
  \begin{itemize}
  \item \package*{kpfonts} qui fournit tous les symboles standard, y compris
    ceux du \Package{amssymb}, mais en propose également de nombreux autres ;
  \item \package*{mathtools} qui lui-même charge \package*{amsmath} (mais en
    l'étendant et en corrigeant certains de ses défauts), si bien que toutes les
    commandes de ces deux packages sont disponibles ;
  \item \package*{mathrsfs} qui fournit de jolies lettres calligraphiques (via
    la commande \docAuxCommand{mathscr}) ;
  \end{itemize}
\item redéfinit certaines commandes et en fournit quelques nouvelles, listées
  ci-dessous ;
\item fournit (en s'appuyant sur les packages \package*{amsthm} et
  \package*{thmtools}) des environnements de type \enquote{théorème} classiques,
  listés ci-dessous.
\end{itemize}

\subsection{Constantes universelles et fonctions courantes}

\begin{docCommand}{I}{}
  Cette commande affiche l'unité imaginaire : \enquote{$\I$}.
\end{docCommand}

\begin{docCommand}{E}{\oarg{argument}}
 Selon que son argument optionnel est utilisé ou pas, cette commande affiche :
  \begin{itemize}
  \item la fonction exponentielle appliquée à l'\meta{argument} ;
  \item le nombre d'Euler \enquote{$\E$}.
  \end{itemize}
\end{docCommand}

\begin{bodycode}[listing and text]
\begin{align}
  \E[\I\pi]+1 & = 0 \\
  \E          & = \sum_{n\geq 0} \frac{1}{n!}
\end{align}
\end{bodycode}

\begin{dbremark}{Syntaxes équivalentes pour la fonction exponentielle}{}
  Il y a équivalence entre les commandes :
  \begin{itemize}
  \item |\E[|\meta{argument}|]| qui est fournie par la classe ;
  \item |\E^{|\meta{argument}|}| qui est plus habituelle.
\end{itemize}
\end{dbremark}

\begin{docCommand}{log(*)}{}
  Cette commande affiche, sous sa forme :
  \begin{description}
  \item[\emph{non} étoilée :] \enquote{$\log$} ;
  \item[étoilée :] \enquote{$\log*$}.
  \end{description}
\end{docCommand}

\begin{docCommand}{lg(*)}{}
  Cette commande affiche, sous sa forme :
  \begin{description}
  \item[\emph{non} étoilée :] \enquote{$\lg$} ;
  \item[étoilée :] \enquote{$\lg*$}.
  \end{description}
\end{docCommand}

\subsection{Ensembles}

\subsection{Ensembles courants}

\begin{docCommand}{bbN}{}
  Cette commande affiche l'ensemble des nombres entiers positifs :
  \enquote{$\bbN$}.
\end{docCommand}

\begin{docCommand}{bbZ}{}
  Cette commande affiche l'ensemble des nombres entiers relatifs :
  \enquote{$\bbZ$}.
\end{docCommand}

\begin{docCommand}{bbD}{}
  Cette commande affiche l'ensemble des nombres décimaux : \enquote{$\bbD$}.
\end{docCommand}

\begin{docCommand}{bbQ}{}
  Cette commande affiche l'ensemble des nombres rationnels : \enquote{$\bbQ$}.
\end{docCommand}

\begin{docCommand}{bbR}{}
  Cette commande affiche l'ensemble des nombres réels : \enquote{$\bbR$}.
\end{docCommand}

\begin{docCommand}{bbC}{}
  Cette commande affiche l'ensemble des nombres complexes : \enquote{$\bbC$}.
\end{docCommand}

\begin{docCommand}{bbK}{}
  Cette commande peut être utilisée pour dénoter un corps quelconque. Elle
  affiche \enquote{$\bbK$}.
\end{docCommand}

\subsection{Ensembles définis}

\begin{docCommand}{set}{\marg{définition}\oarg{caractérisation}}
  Cette commande affiche l'ensemble défini par \meta{définition}, avec une
  éventuelle \meta{caractérisation}\footnote{Partie suivant \enquote{tels
      que}.}.
\begin{bodycode}[listing and text]
\begin{align}
  \bbN & = \set{0,1,2,3,\dots}            \\
  \bbZ & = \bbN\cup\set{-n}[n\in\bbN]     \\
  \bbQ & = \set{p/q}[p\in\bbZ,\ q\in\bbN^*] \\
  \bbQ & = \set{\frac{p}{q}}[p\in\bbZ,\ q\in\bbN^*]
\end{align}
\end{bodycode}
\end{docCommand}

\subsection[Fonctions trigonométriques circulaires et hyperboliques]{Fonctions
  trigonométriques circulaires et hyperboliques, directes et réciproques}

\begin{dbremark}{Commandes propres au français}{}
  Les commandes de cette section ne sont définies (ou redéfinies) que pour les
  articles en français.
\end{dbremark}

\begin{docCommand}{cot}{}
  Cette commande affiche la cotangente : \enquote{$\cot$}. Elle a pour alias
  \docAuxCommand{cotan}.
\end{docCommand}

\begin{docCommand}{arccos}{}
  Cette commande affiche (la détermination principale de) l'arc cosinus :
  \enquote{$\arccos$}.
\end{docCommand}

\begin{docCommand}{arcsin}{}
  Cette commande affiche (la détermination principale de) l'arc sinus :
  \enquote{$\arcsin$}.
\end{docCommand}

\begin{docCommand}{arctan}{}
  Cette commande affiche (la détermination principale de) l'arc tangente :
  \enquote{$\arctan$}.
\end{docCommand}

\begin{docCommand}{cosh}{}
  Cette commande affiche le cosinus hyperbolique : \enquote{$\cosh$}. Elle a pour
  alias \docAuxCommand{ch}.
\end{docCommand}

\begin{docCommand}{sinh}{}
  Cette commande affiche le sinus hyperbolique : \enquote{$\sinh$}. Elle a pour
  alias \docAuxCommand{sh}.
\end{docCommand}

\begin{docCommand}{tanh}{}
  Cette commande affiche la tangente hyperbolique : \enquote{$\tanh$}.
  \begin{dbremark}{Pas de commande \protect\docAuxCommand{th} pour la tangente hyperbolique}{}
    La commande |\th| est déjà définie dans \LaTeX{} et ne peut donc pas être
    utilisée pour la tangente hyperbolique.
  \end{dbremark}
\end{docCommand}

\begin{docCommand}{arccosh}{}
  Cette commande affiche l'argument cosinus hyperbolique :
  \enquote{$\arccosh$}. Elle a pour alias \docAuxCommand{Argch}.
\end{docCommand}

\begin{docCommand}{arcsinh}{}
  Cette commande affiche l'argument sinus hyperbolique :
  \enquote{$\arcsinh$}. Elle a pour alias \docAuxCommand{Argsh}.
\end{docCommand}

\begin{docCommand}{arctanh}{}
  Cette commande affiche l'argument tangente hyperbolique :
  \enquote{$\arctanh$}. Elle a pour alias \docAuxCommand{Argth}.
\end{docCommand}

\subsection{Paires de délimiteurs}

\subsubsection{Commandes prédéfinies}

La classe offre un certain nombre de commandes listées dans le \vref{com-paired}
facilitant la saisie de paires de délimiteurs courants, par exemple pour les
normes, les valeurs absolues, etc.

\begin{table}
  \centering
  \begin{tabular}{ll}
    Commandes               & Exemple           \\\toprule
    \docAuxCommand{norm}    & $\norm{\cdot}$    \\\midrule
    \docAuxCommand{lnorm}   & $\lnorm{\cdot}$   \\\midrule
    \docAuxCommand{llnorm}  & $\llnorm{\cdot}$  \\\midrule
    \docAuxCommand{lpnorm}  & $\lpnorm{\cdot}$  \\\midrule
    \docAuxCommand{supnorm} & $\supnorm{\cdot}$ \\\midrule
    \docAuxCommand{abs}     & $\abs{\cdot}$     \\\midrule
    \docAuxCommand{prt}     & $\prt{\cdot}$     \\\midrule
    \docAuxCommand{brk}     & $\brk{\cdot}$     \\\midrule
    \docAuxCommand{brc}     & $\brc{\cdot}$     \\\midrule
    \docAuxCommand{leqgeq}  & $\leqgeq{\cdot}$  \\\midrule
    \docAuxCommand{lrangle} & $\lrangle{\cdot}$ \\\bottomrule
  \end{tabular}
  \caption{Commandes facilitant la saisie de paires de
    délimiteurs courants}
  \label{com-paired}
\end{table}

Ces commandes ayant toutes la même syntaxe, seule celle de la commande
\docAuxCommand{norm} est détaillée.

\begin{docCommand*}{norm}{\oarg{paramètre de taille}\marg{argument}}
  Cette commande affiche la norme de l'\meta{argument}.

  Par défaut, la hauteur des délimiteurs est automatiquement adaptée à celle de
  l'\meta{argument} mais il est possible de spécifier en argument optionnel un
  \meta{paramètre de taille} :
  \begin{itemize}
  \item \docValue{0} : taille par défaut du délimiteur\footnote{Taille du
      délimiteur en mode texte.} ;
  \item \docValue{1} ou \docAuxCommand{big} : taille du délimiteur plus grande
    que la précédente ;
  \item \docValue{2} ou \docAuxCommand{Big} : taille du délimiteur plus grande
    que la précédente ;
  \item \docValue{3} ou \docAuxCommand{bigg} : taille du délimiteur plus grande
    que la précédente ;
  \item \docValue{4} ou \docAuxCommand{Bigg} : taille du délimiteur plus grande
    que la précédente.
  \end{itemize}
\end{docCommand*}

\begin{docCommand}{norm*}{\marg{argument}}
  Cette commande est équivalente à |\norm[0]{|\meta{argument}|}|.
\end{docCommand}

\begin{bodycode}[listing and text]
\begin{equation}
\supnorm{\frac{A}{2}}^k    \quad
\supnorm[0]{\frac{A}{2}}^k \quad
\supnorm[1]{\frac{A}{2}}^k \quad
\supnorm[2]{\frac{A}{2}}^k \quad
\supnorm[3]{\frac{A}{2}}^k \quad
\supnorm[4]{\frac{A}{2}}^k \quad
\supnorm*{\frac{A}{2}}^k
\end{equation}
\end{bodycode}

\begin{dbremark}{Indices des paires de délimiteurs qui n'en sont pas pourvues par défaut}{}
  Il est toujours possible d'ajouter manuellement un indice à une commande de
  paire de délimiteurs qui n'en est pas pourvue par défaut.
\begin{bodycode}[listing and text]
\begin{equation}
\norm{\frac{A}{2}}_3
\end{equation}
\end{bodycode}
\end{dbremark}

\subsubsection{Commandes à définir}

Il est possible de définir de nouvelles commandes facilitant la saisie de paires
de délimiteurs au moyen de la commande \refCom{NewPairedDelimiter}.
%
\begin{docCommand}{NewPairedDelimiter}{\marg{commande}\marg{options}}
  Cette commande définit une \meta{commande} similaire à la commande
  \docAuxCommand{norm} mais dont les délimiteurs gauche et droit et l'indice
  sont ceux spécifiés en \meta{options} au moyen des clés \refKey{left},
  \refKey{right} et \refKey{subscript}.
  \begin{docKey}{left}{=\meta{délimiteur}}{pas de valeur par défaut,
      initialement vide}
    Cette clé permet de spécifier le délimiteur gauche de la \meta{commande}
    créée par \refCom{NewPairedDelimiter}.
  \end{docKey}
  \begin{docKey}{right}{=\meta{délimiteur}}{pas de valeur par défaut,
      initialement vide}
    Cette clé permet de spécifier le délimiteur droit de la \meta{commande}
    créée par \refCom{NewPairedDelimiter}.
  \end{docKey}
  \begin{docKey}{subscript}{=\meta{indice}}{pas de valeur par défaut,
      initialement vide}
    Cette clé permet de spécifier l'indice de la \meta{commande} créée par
    \refCom{NewPairedDelimiter}.
  \end{docKey}
\end{docCommand}

\NewPairedDelimiter{\floor}{
  left=\lfloor,
  right=\rfloor,
  subscript=\text{eff}
}
\begin{preamblecode}
\NewPairedDelimiter{\floor}{
  left=\lfloor,
  right=\rfloor,
  subscript=\text{eff}
}
\end{preamblecode}
\begin{bodycode}[listing and text]
\begin{equation}
\floor{\frac{A}{2}}
\end{equation}
\end{bodycode}

\subsection{Opérateurs}

\subsubsection{Opérateurs prédéfinis}

\begin{docCommand}{dif}{}
  Cette commande affiche l'opérateur différentiel employé notamment dans les
  intégrales : \enquote{$\dif$}.
\begin{bodycode}[listing and text]
\begin{equation}
\iiint_T f(x,y,z)\dif x \dif y \dif z=0
\end{equation}
\end{bodycode}
\end{docCommand}

\begin{docCommand}{grad}{}
  Cette commande affiche le gradient :
  \begin{itemize}
  \item sous la forme \enquote{$\grad$} par défaut ;
  \item sous la forme \enquote{$\nabla$} si la valeur \docValue{nabla} est
    passée à la clé \refKey{gradient} en argument de la commande
    \refCom{articlesetup}.
  \end{itemize}
  \begin{docKey}{gradient}{=\docValue{grad}\textbar\docValue{nabla}}{pas
      de valeur par défaut, initialement \docValue{grad}}
    Cette clé permet de spécifier la façon dont est affiché le gradient grâce
    à la commande \refCom{grad}.
  \end{docKey}
\end{docCommand}

\begin{docCommand}{Div}{}
  Cette commande affiche la divergence : \enquote{$\Div$}.
  \begin{dbremark}{Pas de commande \protect\docAuxCommand{div} pour la divergence}{}
    La commande |\div| est déjà définie dans \LaTeX{} et ne peut donc pas être
    utilisée pour la divergence.
  \end{dbremark}
\end{docCommand}

\begin{docCommand}{curl}{}
  Cette commande affiche le rotationnel :
  \enquote{$\curl$}. Elle a pour alias \docAuxCommand{rot}.
\end{docCommand}

\begin{bodycode}[listing and text]
\begin{align}
  \Div  F & = \grad \cdot  F \\
  \curl F & = \grad \wedge F
\end{align}
\end{bodycode}

\begin{docCommand}{supp}{}
  Cette commande affiche le support (d'une fonction, d'une mesure, etc.) :
  \enquote{$\supp$}.
\end{docCommand}

\subsubsection{Opérateurs à définir}

Il est possible de définir de nouveaux noms d'opérateurs au moyen de la commande
usuelle \refCom{DeclareMathOperator}.

\begin{docCommand}{DeclareMathOperator}{\marg{commande}\marg{nom}}
  Cette commande définit la \meta{commande} qui affiche l'opérateur nommé
  \meta{nom}.
\end{docCommand}

\begin{preamblecode}
\DeclareMathOperator{\cat}{Cat}
\end{preamblecode}
\begin{bodycode}[listing and text,listing options={deletekeywords={[1]{example}},deletekeywords={[2]{all}}}]
A standard example is $\cat$, the $2$-category of all (small)
categories.
\end{bodycode}

\subsubsection{(Usage avancé) Précautions en cas d'opérations binaires non
  courantes}

Si, dans des opérations binaires, un opérateur\footnote{Prédéfini ou défini au
  moyen de la commande \protect\refCom{DeclareMathOperator}.} est suivi par un
opérateur binaire courant de \LaTeX{} (\docAuxCommand{circ},
\docAuxCommand{cdot}, \docAuxCommand{wedge}, etc.), les espaces horizontales les
séparant seront correctes.

En revanche, si l'opérateur binaire impliqué n'est pas courant, il convient de
le déclarer au moyen de la commande \refCom{BinaryOperators}.

\begin{docCommand}{BinaryOperators}{\marg{opérateurs binaires}}
  Cette commande assure que les \meta{opérateurs binaires}\footnote{Séparés par
    des virgules.} seront, dans des opérations binaires, correctement espacés
  des opérateurs les précédant.
\end{docCommand}

\begin{preamblecode}
\newcommand{\mybinop}{\mathbin{@}}
\end{preamblecode}
\begin{bodycode}[listing and text]
\begin{tabular}{l@{ : }l}
  $\grad \cdot F$    & bon     \\
  $\grad \mybinop F$ & mauvais \\
  \BinaryOperators{\mybinop}%
  $\grad \mybinop F$ & bon (grâce à \verb|\BinaryOperators{\mybinop}|)
\end{tabular}
\end{bodycode}

\subsection{Divers}

\begin{dbremark}{Commandes propres au français (bis)}{}
  Les commandes suivantes ne sont définies redéfinies que pour les articles en
  français.
\end{dbremark}

\begin{docCommand}{leq}{}
  Cette commande affiche \enquote{$\leq$}.
\end{docCommand}
\begin{docCommand}{geq}{}
  Cette commande affiche \enquote{$\geq$}.
\end{docCommand}

\subsection{Environnements de type \enquote{théorème}}

Pour faciliter la composition des théorèmes et objets analogues, la
\nwejmauthorcl{} fournit des environnements prêts à l'emploi et permet d'en
créer de nouveaux si besoin est (cf. commande \refCom{newtheorem}).

\subsubsection{Environnements de type \enquote{théorème} prédéfinis}
\label{sec:envir-de-type}

Le \vref{env-theorems} liste les théorèmes et objets analogues prédéfinis par la
\nwejmauthorcl{} avec :
\begin{description}
\item[en colonne~1] le nom des environnements correspondants ;
\item[en colonne~2] leurs titres récurrents (automatiquement traduits dans la
  langue de l'article).
\end{description}

\begin{table}
  \centering
  \begin{tabular}{ll}
    Environnement                   & Titre (ici en français) \\\toprule
    \docAuxEnvironment{theorem}     & Théorème                \\\midrule
    \docAuxEnvironment{corollary}   & Corollaire              \\\midrule
    \docAuxEnvironment{conjecture}  & Conjecture              \\\midrule
    \docAuxEnvironment{proposition} & Proposition             \\\midrule
    \docAuxEnvironment{lemma}       & Lemme                   \\\midrule
    \docAuxEnvironment{axiom}       & Axiome                  \\\midrule[.75pt]
    \docAuxEnvironment{definition}  & Définition              \\\midrule
    \docAuxEnvironment{remark}      & Remarque                \\\midrule
    \docAuxEnvironment{example}     & Exemple                 \\\midrule
    \docAuxEnvironment{notation}    & Notation                \\\midrule[.75pt]
    \docAuxEnvironment{proof}       & Preuve                  \\\bottomrule
  \end{tabular}
  \caption{Environnements de type \enquote{théorème} fournis}
  \label{env-theorems}
\end{table}

\begin{dbremark}{\enquote{Théorèmes} non numérotés}{}
  Chacun des environnements du \vref{env-theorems} admet une version étoilée
  créant une version non numérotée de l'objet de type \enquote{théorème}
  correspondant.
\end{dbremark}

\begin{bodycode}[listing and text]
\begin{theorem}
  Tout espace métrisable séquentiellement compact est compact.
\end{theorem}
\begin{definition}[congruence modulo $n$]
  Soit $n$ un entier supérieur ou égal à $2$. Deux entiers $a$ et $b$
  sont dits congruents modulo $n$ si $a - b \in n\bbZ$
\end{definition}
\begin{remark*}
  Une des plus belles relations mathématiques est :
  \[
  e^{i\pi}+1=0.
  \]
\end{remark*}
\begin{proof}
  Tout ce qui est rare est cher. Or un cheval bon marché est rare.
  Donc un cheval bon marché est cher.
\end{proof}
\end{bodycode}

On remarque dans l'exemple précédent que les \enquote{théorèmes} ont des mises
en forme variables. La \nwejmauthorcl{} fournit effectivement trois styles de
\enquote{théorèmes}, \docValue{theorem}, \docValue{definition} et
\docValue{proof}, dont les caractéristiques sont indiquées au
\vref{env-theorems-styles}.

\begin{table}
  \centering
  % \setlength{\tymin}{\widthof{romain, terminé}}
  \begin{tabulary}{\linewidth}{lLLL}
Style                 & Titre récurrent                  & Contenu                            & \enquote{Théorèmes} concernés               \\\toprule
\docValue{theorem}    & gras, romain, numéroté           & italique
                      & théorèmes, corollaires, conjectures, propositions, lemmes, axiomes                                                  \\\midrule
\docValue{definition} & gras, romain, numéroté           & romain                             & définitions, remarques, exemples, notations \\\midrule
\docValue{proof}      & non gras, italique, non numéroté & romain, terminé par un carré blanc & preuves                                     \\\bottomrule
  \end{tabulary}
  \caption{Styles de \enquote{théorèmes} fournis}
  \label{env-theorems-styles}
\end{table}

\subsubsection{Environnements de type \enquote{théorème} personnels}
\label{sec:envir-de-type-1}

Si les environnements de \enquote{théorème} fournis par la classe ne suffisent
pas, la commande \refCom{newtheorem} permet d'en créer de nouveaux.

\begin{docCommand}{newtheorem}{\oarg{option(s)}\marg{nom}}
  Cette commande crée un nouvel environnement \LaTeX{}, \meta{nom}, destiné à la
  composition d'un \enquote{théorème} dont, par défaut :
  \begin{itemize}
  \item le titre récurrent est \meta{Nom}\footnote{\Ie*{}, quelle que soit la
      langue de l'article, le \meta{nom} de l'environnement \LaTeX{} dont
      l'initiale est en majuscule.} ;
  \item le style est \docValue{theorem}.
  \end{itemize}
  Cette commande crée également l'environnement étoilé \meta{nom}|*| produisant
  des occurrences \emph{non} numérotées de ce \enquote{théorème}.

  Si les valeurs par défaut du titre récurrent (\meta{Nom}) ou du style
  (\docValue{theorem}) ne conviennent pas, il suffit spécifier celles souhaitées
  en \meta{option(s)} au moyen des clés \refKey{title} et \refKey{style}
  suivantes.
  \begin{docKey}{title}{={\meta{titre récurrent}}}{pas de
      valeur par défaut, initialement vide}
    Cette clé permet de spécifier un \meta{titre récurrent} différent de
    \meta{Nom}, valable quelle que soit la langue de l'article.
  \end{docKey}
  \begin{docKey}{style}{=\docValue{theorem}\textbar\docValue{definition}\textbar\docValue{proof}}{pas
      de valeur par défaut,\\initialement \docValue{theorem}}
    Cette clé permet de spécifier le style du \enquote{théorème} à créer, au
    moyen des valeurs \docValue{theorem}, \docValue{definition} et
    \docValue{proof}.
  \end{docKey}

  Si besoin est, il est même possible de spécifier des titres récurrents
  différents selon la langue utilisée au moyen des clés suivantes :
  \begin{docKey}{title/english}{={\meta{titre récurrent en anglais}}}{pas de
      valeur par défaut,\\initialement vide}
  \end{docKey}
  \begin{docKey}{title/french}{={\meta{titre récurrent en français}}}{pas de
      valeur par défaut,\\initialement vide}
  \end{docKey}
  \begin{docKey}{title/german}{={\meta{titre récurrent en allemand}}}{pas de
      valeur par défaut,\\initialement vide}
  \end{docKey}
  \begin{docKey}{title/dutch}{={\meta{titre récurrent en néerlandais}}}{pas de
      valeur par défaut,\\initialement vide}
    Ces clés permettent de spécifier un \meta{titre récurrent} différent
    de \meta{Nom} pour la langue correspondante.
  \end{docKey}

\begin{preamblecode}[listing options={deletekeywords={[1]{title,subtitle,definition}},morekeywords={[2]{title,subtitle}},morekeywords={[3]{definition}}}]
\newtheorem{article}
\newtheorem[title/french=Fait,style=definition]{fact}
\end{preamblecode}
\begin{bodycode}[listing and text]
\begin{article}[dit premier]
Les hommes naissent et demeurent libres et égaux en droits [...]
\end{article}
\begin{fact*}
Tout ce qui est rare est cher.
\end{fact*}
\begin{fact}
Un cheval bon marché est rare.
\end{fact}
\end{bodycode}
\end{docCommand}

\subsection{Listes d'hypothèses ou d'assertions}
\label{sec:list-dhyp-ou}

Dans les articles en mathématiques, il est fréquent que les auteurs aient
à lister des assertions ou des hypothèses et à s'y référer. Pour ce faire, la
\nwejmauthorcl{} fournit deux environnements qui s'emploient comme l'habituel
environnement \docAuxEnvironment{enumerate} de listes ordonnées (numérotées) :
\refEnv{assertions} et \refEnv{hypotheses}, dans lesquels chaque assertion ou
hypothèse est introduite par la commande \docAuxCommand{item}.

\begin{docEnvironment}[doclang/environment content=assertions]{assertions}{}
  Cet environnement compose une liste d'assertions.
\end{docEnvironment}
\begin{docEnvironment}[doclang/environment content=hypothèses]{hypotheses}{}
  Cet environnement compose une liste d'hypothèses.
\end{docEnvironment}

\begin{bodycode}[listing and text,listing options={deletekeywords={[1]{label}}}]
\begin{axiom}\label{mon-axiom}
  Chaque assertion suivante est admise.
  \begin{assertions}
  \item\label{rare-cher} Tout ce qui est rare est cher.
  \item\label{cheval} Un cheval bon marché est rare.
  \end{assertions}
\end{axiom}
En vertu des \vref{rare-cher,cheval} de l'\vref{mon-axiom},
un cheval bon marché est cher.
\end{bodycode}

\section{Commandes généralistes}
\label{sec:comm-gener}

En plus des commandes propres aux mathématiques, la \nwejmauthorcl{} fournit des
commandes généralistes destinées à faciliter la saisie d'un article à paraître
dans la \nwejm{}.

\begin{docCommand}{ie(*)}{}
  Cette commande affiche l'expression \enquote{\ie*{}} traduite dans la langue
  de l'article, sous sa forme :
  \begin{description}
  \item[\emph{non} étoilée :] en version abrégée \enquote{\ie{}} ;
  \item[étoilée :] en version \emph{non} abrégée \enquote{\ie*{}}.
  \end{description}
\end{docCommand}

\begin{docCommand}{Ie(*)}{}
  Cette commande affiche l'expression \enquote{\Ie*{}} traduite dans la langue
  de l'article, sous sa forme :
  \begin{description}
  \item[\emph{non} étoilée :] en version abrégée \enquote{\Ie{}} ;
  \item[étoilée :] en version \emph{non} abrégée \enquote{\Ie*{}}.
  \end{description}
\end{docCommand}

\begin{docCommand}{century(*)}{\marg{numéro}}
  Cette commande affiche l'ordinal du \meta{numéro}\footnote{Ce numéro doit être
    un nombre entier non nul, le cas échéant négatif.}\footnote{En français et en
    allemand, \meta{numéro} est composé en chiffre romain et en petites
    capitales.} d'un siècle, suivi du mot \enquote{siècle} traduit dans la
  langue de l'article et de la mention éventuelle qu'il s'agit d'un siècle
  précédant notre ère. Ainsi :
\begin{bodycode}[listing and text]
Au \century{-1}, [...]. Plus tard, au \century{8}, [...]
\end{bodycode}
  Sa version étoilée n'affiche que l'ordinal (sauf si le \meta{numéro} est
  négatif). Ainsi :
\begin{bodycode}[listing and text]
L'agriculture au \century*{-1} était [...].
La poésie du \century*{19} aura marqué [...]
\end{bodycode}
\end{docCommand}

\begin{docCommand}{aside(*)}{\marg{texte}}
  Cette commande permet de composer du \meta{texte} entre tirets longs, pour une
  \enquote{incise} (un peu à la manière de parenthèses).
  \begin{description}
  \item[Sous sa forme non étoilée,] elle insère le \meta{texte} entre deux
    tirets longs.
  \item[Sous sa forme étoilée,] elle fait (seulement) précéder le \meta{texte}
    d'un tiret, ce qui est à utiliser en fin de phrase.
  \end{description}
  Ainsi :
\begin{bodycode}[listing and text]
Des expériences \aside{dans le monde \enquote{réel}} ont même été
suscitées par des expériences numériques.
\end{bodycode}
  et :
\begin{bodycode}[listing and text,listing options={deletekeywords={exemple}}]
On soupçonne que non \aside*{par exemple on s'attend à ce que
  $1/\pi$ ne soit pas une période}.
\end{bodycode}
\end{docCommand}

\begin{docCommand}{nwejm}{}
  Cette commande affiche, sous sa forme :
  \begin{description}
  \item[\emph{non} étoilée :] la version abrégée \enquote{\nwejm{}} ;
  \item[étoilée :] la version \emph{non} abrégée \enquote{\nwejm*{}}.
  \end{description}
\end{docCommand}

\section{Sélection d'outils fournis par des packages tiers}

Cette section donne une liste d'outils (commandes et environnements) fournis par
des packages automatiquement chargés par la \nwejmauthorcl\footnote{La
  \vref{sec-packages-charges-ou-pas} liste ceux dont les fonctionnalités peuvent
  être utiles aux auteurs.}.

\begin{dbwarning}{Liste des fonctionnalités des packages tiers non exhaustive}{}
  Cette liste est \emph{non exhaustive} :
  \begin{itemize}
  \item seule une infime partie des packages existants est chargée par la
    classe ;
  \item seuls certains packages chargés par la classe sont cités ;
  \item seules certaines fonctionnalités des packages cités sont décrites. Le
    lecteur souhaitant en savoir davantage est invité à lire la documentation de
    ces packages.
  \end{itemize}
\end{dbwarning}

\subsection{Références croisées}
\label{sec-references-croisees}

Le \Package*{cleveref}, utilisé conjointement avec le \Package*{varioref}, rend la
commande \refCom{vref} de ce dernier très puissante.

\begin{docCommand}{vref}{\marg{label}}
  Cette commande affiche le numéro d'un objet\footnote{Section, équation,
    théorème, figure, tableau, etc.} labellisé par la commande
  \lstinline[deletekeywords={label}]+\label{+\meta{label}|}| et :
  \begin{itemize}
  \item affiche (ou pas selon le contexte) le numéro de la page où se situe cet
    objet\footnote{Fonctionnalité fournie par le \Package{varioref}.} ;
  \item détecte la nature de cet objet et fait précéder le numéro de celui-ci du
    mot clé correspondant\footnote{Fonctionnalité fournie par le
      \Package{cleveref}.} automatiquement traduit dans la langue de
    l'article\footnote{Fonctionnalité fournie par la \nwejmauthorcl.} ;
  \item fait de la référence à l'objet (numéro et éventuel numéro de page) un
    lien hypertexte vers l'objet\footnote{Fonctionnalité fournie par le
      \Package{hyperref}.}.
  \end{itemize}
\begin{bodycode}[listing and text]
[...] on se reportera à la \vref{sec-canevas-de-fichier} pour un
canevas de fichier source [...]
\end{bodycode}
\end{docCommand}

% \begin{dbremark}{\docAuxCommand{ref} et \docAuxCommand{eqref} : clones de \protect\refCom{vref}}{}
%   Les traditionnelles commandes \docAuxCommand{ref} et \docAuxCommand{eqref} ont
%   été redéfinies par la \nwejmauthorcl pour fonctionner de façon identique
%   à \refCom{vref}.
% \end{dbremark}

\subsection{\enquote{Équations} (non-)numérotées}
\label{sec:equat-non-numer}

Pour les \enquote{équations}, le \Package*{amsmath} fournit les environnements
mathématiques :
\begin{itemize}
\item |equation|, |align|, |gather|, |multline| et |flalign| qui numérotent
  leur contenu ;
\item |equation*|, |align*|, |gather*|, |multline*| et |flalign*| qui \emph{ne}
  numérotent \emph{pas} leur contenu ;
\end{itemize}
les premiers étant à utiliser si (et seulement si, théoriquement) les contenus
correspondants font l'objet de références croisées.

Avoir à choisir entre les versions étoilées ou pas présente au moins deux
inconvénients pour l'utilisateur :
\begin{enumerate}
\item cela l'oblige à se poser à chaque fois la question de la
  numérotation de son équation ;
\item son choix peut ultérieurement s'avérer caduc\footnote{Et donc nécessiter
    de passer par exemple d'un environnement étoilé à un non-étoilé \aside*{donc
      avec ajout d'un label}.}
\end{enumerate}
Pour pallier cela, le \Package{autonum} :
\begin{enumerate}
\item permet de ne recourir qu'aux environnements non-étoilés ;
\item assure automatiquement que les \enquote{équations} effectivement
  numérotées seront celles, et seulement celles, s'avérant faire l'objet de
  références croisées au sein du document.
\end{enumerate}

\begin{bodycode}[listing and text,listing options={deletekeywords={label}}]
L'\vref{euler-identy} mérite une référence croisée.
\begin{equation}\label{euler-identy}
  \E[\I\pi]+1=0
\end{equation}
C'est bien la seule ici.
\begin{equation}\label{not-interesting}
  0=0
\end{equation}
Malheureusement.
\begin{equation}
  1=1
\end{equation}
\end{bodycode}

\subsection{Acronymes}
\label{sec:acronymes}

Il est fréquent que des acronymes doivent figurer dans un article de
mathématiques et le \Package{glossaries} propose pour cela une fonctionnalité
à la fois très efficace et très simple : il suffit de recourir à ses commandes :
\begin{description}
\item[\refCom{newacronym}] pour \emph{définir} un acronyme :
\item[\refCom{gls} (ou \refCom{acrshort})] pour \emph{afficher} un acronyme.
\end{description}

\begin{docCommand}{newacronym}{\marg{clé}\marg{forme courte}\marg{forme longue}}
  Cette commande, à utiliser de préférence en préambule, définit un acronyme
  où :
  \begin{enumerate}
  \item \meta{clé} identifie l'acronyme de manière unique dans le
    document\footnote{À l'auteur de veiller à ne pas utiliser deux fois la même
      clé pour identifier des acronymes distincts.} ;
  \item \meta{forme courte} est l'acronyme en lui-même ;
  \item \meta{forme longue} est la signification de l'acronyme.
  \end{enumerate}
\begin{dbwarning}{Forme courte des acronymes : en bas de casse}{}
  La \meta{forme courte} d'un acronyme est à saisir exclusivement en bas de
  casse (en minuscules) car elle sera en fait composée en petites capitales.
\end{dbwarning}
\end{docCommand}

\begin{docCommand}{gls}{\marg{clé}}
  Cette commande affiche l'acronyme identifié par \marg{clé} selon le principe
  suivant :
  \begin{enumerate}
  \item la 1\iere{} occurrence de cette commande dans le document affiche
    l'acronyme sous sa forme \emph{complète}, \ie*{} sa \meta{forme longue}
    suivie de sa \meta{forme courte} entre parenthèses ;
  \item les occurrences suivantes affichent l'acronyme uniquement sous
    \meta{forme courte}.
  \end{enumerate}
\end{docCommand}

\begin{preamblecode}
\newacronym{bap}{bap}{bounded approximation property}
\end{preamblecode}
\begin{bodycode}[listing and text]
\begin{enumerate}
\item \gls{bap},
\item \gls{bap}.
\end{enumerate}
\end{bodycode}

\begin{docCommand}{acrshort}{\marg{clé}}
  Cette commande affiche la \meta{forme courte} (seule) de l'acronyme, quel que
  soit le contexte\footnote{\Ie*{} même si c'est la 1\iere{} fois que l'acronyme
    est utilisé dans le document.}.
\end{docCommand}

\begin{dbremark}{Commande \refCom{acrshort} : utile dans un titre}{}
  La commande \refCom{acrshort} peut être utile notamment dans le titre d'un
  article où l'on ne souhaite pas que soit détaillée la forme complète d'un
  acronyme.
\end{dbremark}

Le \Package*{glossaries} fournit bien d'autres commandes et offre bien d'autres
fonctionnalités\autocite[Pour plus de détails, cf. par exemple][]{Bitouze7}.

\subsection{Guillemets, citations}
\label{sec-guillemets-citations}

Le \Package*{csquotes} est dédié aux citations, formelles et informelles,
d'extraits de textes. Il propose notamment la commande \refCom{enquote}
suivante, simple d'emploi.

\begin{docCommand}{enquote}{\marg{texte}}
  Cette commande compose le \meta{texte} entre guillemets, automatiquement
  adaptés :
  \begin{itemize}
  \item aux standards typographiques de la langue en cours. On pourra donc
    l'utiliser dès que du texte est à mettre entre guillemets ;
  \item au niveau (1 ou 2) de \enquote{citation} en cas d'emboîtement. On
    pourra donc également l'utiliser dès que du texte est à citer de façon
    informelle.
  \end{itemize}
\begin{bodycode}[listing and text]
Il lui répondit : \enquote{Courteline disait: \enquote{Passer
    pour un idiot aux yeux d'un imbécile est une volupté de fin
    gourmet.}}
\end{bodycode}
\end{docCommand}

\begin{docCommand}{blockcquote}{\marg{clé}\marg{extrait}}
  Cette commande permet de citer un \meta{extrait} de façon
  formelle\footnote{\Ie*{} avec précision de la source.}, la référence
  bibliographique correspondante devant figurer dans un des fichiers \file{.bib}
  ajoutés\footnote{Au moyen de la commande \docAuxCommand{addbibresource}} et
  y étant identifiée par la clé \meta{clé}.

\begin{bodycode}[listing and text]
\citeauthor{Bitouze} précise que :
\blockcquote{Bitouze}{%
  \textins{Cette} commande compose les citations en détachant
  automatiquement du paragraphe courant celles qui sont
  longues\footnote{Par défaut, ce sont celles
    dépassant 3~lignes.}.%
}.
\end{bodycode}
\end{docCommand}

La commande \refCom{blockcquote} admet des arguments optionnels permettant
d'ajouter du texte \meta{antérieur} et/ou \meta{postérieur} à la citation
d'extrait\autocite[Pour plus de détails, cf. par exemple][]{Bitouze}.

% (\textsc{À suivre...})

\subsection{Adresses Web (\texorpdfstring{\acrshort{url}}{URL})}
\label{sec-url}

Le \Package*{hyperref} fournit (entre autres) la commande \refCom{url} qui
permet de facilement afficher des adresses Web, aussi appelées
\enquote{\gls{url}}, même si elles comportent des caractères spéciaux de \TeX{}
(|#|,
%
\lstinline[commentstyle={}]+%+,
%
|_|, |~|, |&|, etc.) : ces caractères peuvent donc être saisis tels quels, sauf
si la commande \refCom{url} est utilisée dans l'argument d'une autre
commande\footnote{Par exemple \protect\docAuxCommand*{foonote}.} auquel cas les
caractères |#| et
%
\lstinline[commentstyle={}]+%+
%
sont alors à faire précéder de la commande \enquote{d'échappement} |\|
(contre-oblique).%

\begin{docCommand}{url}{\marg{adresse Web}}
  Cette commande affiche l'\meta{adresse Web} saisie \emph{telle quelle} et fait
  d'elle un lien hypertexte.
\begin{bodycode}[listing and text,listing options={deletekeywords={url,math,nwejm,exemple,pdf}}]
Nous renvoyons aux instructions aux auteurs sur notre site Web
\url{http://math.univ-lille1.fr/~nwejm/#Authors}.
\end{bodycode}
\end{docCommand}

\subsection{Listes en ligne}
\label{sec:listes-en-ligne}

Le lecteur est certainement familier des environnements
\docAuxEnvironment{itemize}, \docAuxEnvironment{enumerate} et
\docAuxEnvironment{description} permettant de créer des listes respectivement
non ordonnées\footnote{\enquote{À puces}.}, ordonnées\footnote{\Ie*{}
  numérotées} et \enquote{de description}. Le \Package*{enumitem} fournit des
versions étoilées de ces environnements qui composent ces listes \enquote{en
  ligne}, c'est-à-dire au sein d'un même paragraphe\footnote{Le but premier de
  ce package est de personnaliser les listes mais cela est déconseillé dans le
  cadre du \nwejm{} car cela pourrait contrevenir à la marche typographique
  qu'il a adoptée.}.

\begin{docEnvironment}[doclang/environment content=liste]{itemize*}{}
  Cet environnement compose une liste \enquote{à puces} en ligne.
\end{docEnvironment}
\begin{docEnvironment}[doclang/environment content=liste]{enumerate*}{}
  Cet environnement compose une liste \enquote{numérotée} en ligne.
\end{docEnvironment}
\begin{docEnvironment}[doclang/environment content=liste]{description*}{}
  Cet environnement compose une liste \enquote{de description} en ligne.
\end{docEnvironment}

\begin{bodycode}[listing and text,listing options={deletekeywords={url,math,url,pages}}]
\begin{enumerate}
\item Les données que j’ai rassemblées [...]
\item Les données que j’ai recueillies sont
  \begin{enumerate*}
  \item disponibles publiquement sur internet : pages Web et CV des
    conférenciers ;
  \item accessibles par Wikipédia, par le précieux
    \enquote{mathematics genealogy project}
    (\url{http://genealogy.math.ndsu.nodak.edu/}) ;
  \item pour 60 des conférenciers pour lesquelles mes informations
    étaient trop lacunaires, [...].
  \end{enumerate*}
\end{enumerate}
\end{bodycode}

\appendix

\section{Canevas de fichier source typique}
\label{sec-canevas-de-fichier}

Le lecteur souhaitant écrire un article pour le \nwejm*{} trouvera ci-dessous un
canevas de fichier source qui est un modèle typique d'article standard où les
commandes et environnement moins fréquents ont été mis en commentaire. Il est
attaché au fichier \file{PDF} de la présente documentation\footnote{Pour
  l'extraire, il devrait suffire de cliquer sur l'icône ci-dessous. Il peut
  également être copié-collé : la plupart des afficheurs \textsc{pdf}
  permettent de sélectionner et de copier du texte.}.
%
\newcommand{\attachfilename}{template.tex}%
\newcommand{\attachfiletextlink}{Canevas de fichier source typique}%
\tcbinputlisting{%
  samples,
  fontupper=\scriptsize\ttfamily,
  listing only,
  listing file=../../examples/\attachfilename,
  title={\attachfiletextlink\ \attachfile{../../examples/\attachfilename}},%
  listing options={deletekeywords={title,subtitle,author,keywords},deletekeywords={[2]{first}}}
}

\section{Packages chargés (ou pas) par la
  classe}\label{sec-packages-charges-ou-pas}

\subsection{Packages chargés par la classe}\label{sec:packages-charges-par}

On a vu que, pour plusieurs de ses fonctionnalités, la \nwejmauthorcl{} s'appuie
sur des packages qu'elle charge automatiquement. Ceux dont les fonctionnalités
peuvent être utiles aux auteurs d'articles du \nwejm{} sont répertoriés dans la
liste suivante qui indique leur fonction et le cas échéant la ou les options
avec lesquelles ils sont chargés.

En sus des outils propres à la \nwejmauthorcl, tous ceux fournis par ces
différents packages sont donc à disposition des auteurs de la \nwejm{}.

\begin{description}
\item[\package{nag} :] signalement\footnote{Sous forme de \emph{warnings}.} de
  packages, commandes et environnements obsolètes :
  \begin{description}
  \item[options par défaut :] \docAuxKey{l2tabu}, \docAuxKey{orthodox} ;
  \end{description}
\item[\package{kpfonts} :] police principale du document :
  \begin{description}
  \item[option par défaut :] \docAuxKey{noDcommand} ;
  \end{description}
\item[\package{graphicx} :] inclusion d'images ;
\item[\package{adjustbox} :] ajustement de la position de boîtes, par exemple
  d'images ;
\item[\package{xspace} :] définition de commandes qui ne \enquote{mangent} pas
  l'espace qui suit ;
\item[\package{array} :] extension (et corrections de bogues) des
  environnements de tableaux ;
\item[\package{booktabs} :] tableaux d'allure professionnelle ;
\item[\package{csquotes} :] citations d'extraits informelles et
  formelles\footnote{Avec citation des sources,
    cf. \vref{sec-guillemets-citations}.} :
    \begin{description}
    \item[option par défaut :] \docAuxKey{autostyle} ;
    \item[réglage par défaut] |\SetCiteCommand{\autocite}| ;
  \end{description}
\item[\package{biblatex} :] gestion puissante de la bibliographie ;
\item[\package{datetime2} :] formats de dates et de (zones de) temps :
  \begin{description}
  \item[option par défaut :] \docAuxKey{useregional} ;
  \end{description}
\item[\package{hyperref} :] support pour les liens
  hypertextes\footnote{Cf. \vref{sec-url}.} :
  \begin{description}
  \item[option par défaut :] \docAuxKey{hidelinks},
    \docAuxKey{hypertexnames}(|=false|) ;
  \end{description}
\item[\package{glossaries} :] création de glossaires et (listes d')acronymes :
  \begin{description}
  \item[option par défaut :] \docAuxKey{nowarn} ;
  \end{description}
\item[\package{varioref} et \package{cleveref} :] références croisées
  intelligentes\footnote{Cf. \vref{sec-references-croisees}.}.
\end{description}

\subsection{Packages non  chargés par la classe}\label{sec:packages-non-charges}

La liste suivante, loin d'être exhaustive, répertorie des packages non chargés
par la \nwejmauthorcl{} mais pouvant se révéler utiles aux auteurs.  En outre,
lorsqu'ils sont chargés manuellement par l'utilisateur, certains d'entre eux se
voient fixés par la \nwejmauthorcl{} des options ou réglages dont les plus
notables sont précisés.

\begin{description}
\item[\package{tikz-cd} :] création simple de diagrammes commutatifs de très
  haute qualité\footnote{Le \Package{xy}, souvent utilisé pour de tels
    diagrammes, est incomptible avec la présente :
    cf. \vref{sec-incompatibilites}.} ;
\item[\package{pgfplots} :] création simple de graphiques (de dimensions~$2$
  ou~$3$) de très haute qualité pour représenter des fonctions ou des données
  expérimentales ;
\item[\package{siunitx} :] gestion des nombres, angles et unités, et alignement
  vertical sur le séparateur décimal dans les tableaux :
  \begin{description}
  \item[option par défaut :]\
    \begin{itemize}
    \item \docAuxKey{detect-all} ;
    \item \docAuxKey{locale}|=|\docValue{FR} ou \docValue{UK} ou \docValue{DE}
      selon la langue de l'article ;
    \end{itemize}
  \end{description}
\item[\package{listings} :] insertion de listings informatiques ;
\item[\package{todonotes} :] insertion de \enquote{TODO}\footnote{Rappels de
    points qu'il ne faut pas oublier d'ajouter, de compléter, de réviser, etc.}.
\end{description}

\section{Incompatibilités}
\label{sec-incompatibilites}

Pour des raison techniques, le \nwejm{} n'accepte pas de fichier source
\file{.tex} chargeant le \Package{xy} et la classe \nwejmauthor{} émet une
erreur si tel est le cas. L'utilisateur ayant à composer des diagrammes
\enquote{de type matriciel}, notamment des diagrammes commutatifs, est invité
à recourir au \Package{tikz-cd} à la fois moderne et convivial.

\section{Notations, syntaxe, terminologie et codes couleurs}\label{sec-synt-term-notat}

Nous précisons ici les notations, syntaxe, terminologie et codes couleurs de
la présente documentation.

\subsection{Commandes, environnements, clés, valeurs}\label{sec-comm-envir-cles}

Les commandes, environnements, clés et valeurs de clés sont systématiquement
composés en fonte à chasse fixe. En outre, pour plus facilement les
distinguer, ils figurent avec des couleurs propres :
\begin{itemize}
\item les commandes en bleu : \docAuxCommand*{commande} ;
\item les environnements en \enquote{sarcelle} :
  \docAuxEnvironment*{environnement} ;
\item les clés en pourpre : \docAuxKey*{clé} ;
\item les valeurs des clés en violet : \docValue*{valeur}.
\end{itemize}

\subsection{Arguments génériques}
\label{sec-arguments-generiques}

Pour expliquer le rôle d'une commande, il est parfois nécessaire d'indiquer
à quoi celle-ci s'applique, autrement dit quel en est l'argument générique.
Un tel argument est composé :
\begin{itemize}
\item en fonte à chasse fixe ;
\item en italique ;
\item entre chevrons simples ;
\end{itemize}
le tout en marron, ainsi : \meta{argument générique}.

\subsection{Liens hypertextes}
\label{sec-liens-hypertextes}

Les liens hypertextes figurent en couleur, ainsi :
\href{http://gte.univ-littoral.fr/members/dbitouze/pub/latex}{lien hypertexte}.
La plupart des références aux commandes, environnements et clés définis dans la
présente documentation, sont des liens hypertextes, surmontés du numéro de page
où se trouve la cible correspondante (sauf si elle se situe sur la même page) :
\begin{itemize}
\item \refCom{author} ;
\item \refEnv{abstract}.
\end{itemize}


\subsection{Éléments \enquote{obligatoires}}
\label{sec-comm-oblig}

L'icône \mandatory{}, figurant en regard de certains éléments (commandes ou
environnements), indique que ceux-ci sont \enquote{obligatoires}.

\subsection{Codes sources}
\label{sec-codes-sources}

Les exemples qui illustrent la présente documentation sont constitués de codes
sources et, le cas échéant, des \enquote{copies d'écran} correspondantes.
% Ceux-ci proviennent le plus souvent du spécimen de document composé avec la
% \yatcl, fourni avec l'ensemble de la classe
% (cf. \vref{sec-specimen-arborescence}).

Ces codes sources figurent dans des cadres, pouvant comporter un titre, de
couleur bleu :
\begin{itemize}
\item non ombrés s'ils doivent être saisis dans le corps du document ;
\begin{bodycode}
"\meta{code source}"
\end{bodycode}
\begin{bodycode}[title=\meta{titre}]
"\meta{code source}"
\end{bodycode}
\item ombrés s'ils doivent être saisis en préambule du fichier.
\begin{preamblecode}
"\meta{code source à insérer en préambule}"
\end{preamblecode}
\begin{preamblecode}[title=\meta{titre}]
"\meta{code source à insérer en préambule}"
\end{preamblecode}
\end{itemize}

\subsection{Espaces dans les codes sources}
\label{sec-espaces-dans-les}

Pour éviter certaines confusions, les espaces dans les codes sources devant
être saisis au clavier sont parfois matérialisés au moyen de la marque
\lstinline[showspaces]+ +.

\subsection{Options}
\label{sec-options}

La présente classe, ainsi que certaines de ses commandes et certains de ses
environnements, peuvent être modulés au moyen d'options, ou listes d'options
(séparées par des virgules). Ces options peuvent se présenter sous la forme
\meta{clé}|=|\meta{valeur} et la \meta{valeur} passée à une \meta{clé} peut
être :
%^^A \begin{description}
%^^A \item[ne prennent pas de valeur.] Une telle option, par exemple nommée
%^^A   \refKey{option}, est alors documentée selon la syntaxe suivante:
%^^A     \begin{docKey*}{option}{}{\meta{valeurs par défaut et initiale}}
%^^A       \meta{Description de \refKey{option}}
%^^A     \end{docKey*}
%^^A \item[prennent des valeurs.] Une telle option se présente alors sous la forme
%^^A   \meta{clé}"="\meta{valeur}. Les valeurs passées à une clé peuvent être :
\begin{description}
\item[libre.] Si une telle \meta{clé} est (pour l'exemple) nommée
  \refKey{freekey}, elle est alors documentée selon la syntaxe suivante :
  \begin{docKey*}{freekey}{=\meta{valeur}}{\meta{valeurs par défaut et initiale}}
    \meta{Description de \refKey{freekey}}
  \end{docKey*}
\item[imposée] (parmi une liste de valeurs possibles). Si une telle \meta{clé} est
   (pour l'exemple) nommée \refKey{choicekey} et de valeurs imposées
  \docValue*{valeur1}, \dots{}, \docValue*{valeurN}, elle est alors
  documentée selon la syntaxe suivante\footnote{Comme souvent en informatique,
    la barre verticale séparant les valeurs possibles signifie \enquote{ou}.} :
  \begin{docKey*}{choicekey}{=\docValue*{valeur1}\textbar...\textbar\docValue*{valeurN}}{\meta{valeurs par défaut et initiale}}
    \meta{Description de \refKey{choicekey} et de ses valeurs possibles}
  \end{docKey*}
\end{description}
%^^A \end{description}

Les \meta{valeurs par défaut et initiale} d'une clé sont souvent précisées
(entre parenthèses en fin de ligne). Elles indiquent ce que la clé vaut :
\begin{description}
\item[par défaut] c'est-à-dire lorsque la clé \emph{est} employée, mais
  \emph{seule} c'est-à-dire sans qu'une valeur explicite lui soit passée ;
\item[initialement] c'est-à-dire lorsque la clé \emph{n'est pas} employée.
\end{description}
%
\printbibliography
%
% \tcblistof{dbwarninglist}{Table des
%   avertissements\addcontentsline{toc}{section}{Table des avertissements}}%
% \tcblistof{dbremarklist}{Table des remarques\addcontentsline{toc}{section}{Table
%     des remarques}}%
% \tcblistof[\chapter*]{dbexamplelist}{Table des exemples\addcontentsline{toc}{chapter}{Table des exemples}}%
% \tcblistof[\chapter*]{dbfaqlist}{Table des
% questions\addcontentsline{toc}{chapter}{Table des questions}
%
\printindex
\end{document}

%%% Local Variables:
%%% mode: latex
%%% TeX-master: t
%%% End:
