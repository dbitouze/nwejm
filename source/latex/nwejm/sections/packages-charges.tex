\subsection{Packages chargés (ou pas) par la
  classe}\label{sec-packages-charges-ou-pas}

\subsubsection{Packages chargés par la classe}\label{sec:packages-charges-par}

On a vu que, pour plusieurs de ses fonctionnalités, la \nwejmauthorcl{} s'appuie
sur des packages qu'elle charge automatiquement. Ceux dont les fonctionnalités
peuvent être utiles aux auteurs d'articles du \nwejm{} sont répertoriés dans la
liste suivante qui indique leur fonction et le cas échéant la ou les options
avec lesquelles ils sont chargés.

En sus des outils propres à la \nwejmauthorcl, tous ceux fournis par ces
différents packages sont donc à disposition des auteurs de la \nwejm{}.

\begin{description}
\item[\package{nag} :] signalement\footnote{Sous forme de \emph{warnings}.} de
  packages, commandes et environnements obsolètes :
  \begin{description}
  \item[options par défaut :] \docAuxKey{l2tabu}, \docAuxKey{orthodox} ;
  \end{description}
\item[\package{kpfonts} :] police principale du document :
  \begin{description}
  \item[option par défaut :] \docAuxKey{noDcommand} ;
  \end{description}
\item[\package{graphicx} :] inclusion d'images ;
\item[\package{adjustbox} :] ajustement de la position de boîtes, par exemple
  d'images ;
\item[\package{xspace} :] définition de commandes qui ne \enquote{mangent} pas
  l'espace qui suit ;
\item[\package{array} :] extension (et corrections de bogues) des
  environnements de tableaux ;
\item[\package{booktabs} :] tableaux d'allure professionnelle ;
\item[\package{csquotes} :] citations d'extraits informelles et
  formelles\footnote{Avec citation des sources,
    cf. \vref{sec-guillemets-citations}.} :
    \begin{description}
    \item[option par défaut :] \docAuxKey{autostyle} ;
    \item[réglage par défaut] |\SetCiteCommand{\autocite}| ;
  \end{description}
\item[\package{biblatex} :] gestion puissante de la bibliographie ;
\item[\package{datetime2} :] formats de dates et de (zones de) temps :
  \begin{description}
  \item[option par défaut :] \docAuxKey{useregional} ;
  \end{description}
\item[\package{hyperref} :] support pour les liens
  hypertextes\footnote{Cf. \vref{sec-url}.} :
  \begin{description}
  \item[option par défaut :] \docAuxKey{hidelinks},
    \docAuxKey{hypertexnames}(|=false|) ;
  \end{description}
\item[\package{glossaries} :] création de glossaires et (listes d')acronymes :
  \begin{description}
  \item[option par défaut :] \docAuxKey{nowarn} ;
  \end{description}
\item[\package{varioref} et \package{cleveref} :] références croisées
  intelligentes\footnote{Cf. \vref{sec-references-croisees}.}.
\end{description}

\subsubsection{Packages non  chargés par la classe}\label{sec:packages-non-charges}

La liste suivante, loin d'être exhaustive, répertorie des packages non chargés
par la \nwejmauthorcl{} mais pouvant se révéler utiles aux auteurs.  En outre,
lorsqu'ils sont chargés manuellement par l'utilisateur, certains d'entre eux se
voient fixés par la \nwejmauthorcl{} des options ou réglages dont les plus
notables sont précisés.

\begin{description}
\item[\package{tikz-cd} :] création simple de diagrammes commutatifs de très
  haute qualité\footnote{Et offrant une syntaxe plus naturelle que le
    \Package*{xymatrix}.} ;
\item[\package{pgfplots} :] création simple de graphiques (de dimensions~$2$
  ou~$3$) de très haute qualité pour représenter des fonctions ou des données
  expérimentales ;
\item[\package{siunitx} :] gestion des nombres, angles et unités, et alignement
  vertical sur le séparateur décimal dans les tableaux :
  \begin{description}
  \item[option par défaut :]\
    \begin{itemize}
    \item \docAuxKey{detect-all} ;
    \item \docAuxKey{locale}|=|\docValue{FR} ou \docValue{UK} ou \docValue{DE}
      selon la langue de l'article ;
    \end{itemize}
  \end{description}
\item[\package{listings} :] insertion de listings informatiques ;
\item[\package{todonotes} :] insertion de
  \enquote{english}{TODOs}\footnote{Rappels de points qu'il ne faut pas oublier
    d'ajouter, de compléter, de réviser, etc.}.
\end{description}

%%% Local Variables:
%%% mode: latex
%%% TeX-master: "../nwejmdoc.tex"
%%% End:
